\vspace{5em}


\begin{adjustwidth}{-5em}{5em}
\begin{large}
\begin{center}

The Hidden Costs of Decisions in Applied Microeconomics \\
Dissertation Summary\\[2em]

\end{center}
\end{large}
\textit{Chapter 1: College Alcohol Consumption and NCAA Tournament Participation: the Health Cost of March Madness}\\

Using the Harvard School of Public Health College Alcohol Study (CAS), we examine the effect of the NCAA Tournament on the level of binge drinking across a nationally representative survey of American campuses. Colleges and universities across the United States continue to increase expenditure on intercollegiate athletics. A focus on athletics may augment visibility of the university to prospective students and thereby benefit the school. It may also have a negative effect on the current student body by influencing risky behavior related to intercollegiate athletics, especially the consumption of alcohol commonly associated with game day festivities. We find that participation in the NCAA Tournament is associated with a 28\% increase in binge drinking by male students at participating schools. These results suggest that in addition to enjoying the benefits of greater athletic success, universities should confront the culture of binge drinking on game days by seeking to reduce binge drinking during the tournament, particularly among male students.\\

\textit{Chapter 2: Agency Theory and the Decision to Work from Home}\\

Applying the principal-agent model to the wage penalties experienced by individuals working from home, I predict that wages will converge between home-workers and office-workers as monitoring costs fall over time. I also predict that wage variance will decline for home-workers as monitoring costs fall. The nature of working from home has changed quickly as improved communications technology has made it easier for workers to correspond quickly and efficiently from home. Using data from the American Community Survey and Census data, I find that the wage penalty for home-workers fell from around 30\% in 1980 to a wage premium of 1.5\% in 2013 using a selection model to account for potential selection bias. Additionally, I find that wage variance for home-workers fell from 75\% higher in 1980 to 8.5\% higher relative to office-workers in 2013. This strongly suggests that the principal-agent model is appropriate in explaining changes in the nature of working from home.\\[3em]

\textit{Chapter 3: Network Externalities and Friendly Neighbors: When Firms Choose to Invite Competition}\\

In this paper, we propose that under certain circumstances firms may instead choose to \emph{reduce} barriers to entry as a profit-maximizing mechanism so long as the firm is able to induce growth in demand that exceeds the growth in supply provided by new entrants to the market. Economic theory on the subject of barriers to entry focuses almost exclusively on firms working to preserve market power and economic profits. We model the conditions under which this effect may exist and predict that in some industries, an increase in the number of participating firms will induce enough growth in the industry to allow existing firms to increase profit by inducing other firms to enter the market. We show that firms can in fact reduce barriers to entry for competitors and experience increased profits. This model differs from the agglomeration literature by proposing that firms deliberately lower fixed costs for their competitors as a rational act, instead of simply existing in locations where fixed costs are reduced by concentration of firms.

\end{adjustwidth}